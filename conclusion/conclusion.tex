\chapter{Conclusions}
\dsp
\justifying
GO's metastability is a clear hurdle for researchers who want to reach a common vision of this nanomaterial. The plethora of GO configurations prevents researchers from reaching an agreement on underlying basic principles in the field, thus creating bottlenecks that prevent further progress.

This dissertation attempts to build a solid foundation in the field to address and resolve some of the debates that have arisen in the last five years. In particular, this work offers:
\begin{itemize}
\item A standardization of GO offering high-throughput characterization techniques and clustering algorithms that are able to identify six classes of GO based on its properties. The experimental work shows that applications based on GO from the same class tend to have the same macroscopic performance. The GO classification scheme aims to support researchers by enabling a fair comparison among scientific investigations.
\item The resolution of water transport in GO through novel Lattice-Boltzmann simulations, which highlight the friction-like water transport inside GO nanochannels. The simulations are supported by experimental results that identify separation distance between neighboring flakes as the dominant property dictating water transport inside GO.
\item Evidence of the ability to fine-tune GO properties by creating GO nanoarchitectures: i) graphene oxide nanoscrolls (GONS) whose dimensions can be regulated via ultrasound power; ii) graphene oxide architectural laminate (GOAL) membrane whose water transport can be controlled by chemically modifying the GO surface functionality; iii) fully carbon membrane (FCM) showing the potential of GO in nanofiltration applications for water reclamation processes.
\end{itemize}

For future investigations, we establish a line of research aimed to elucidate GO nanotoxicity through ingestion exposure. This methodology can be extended to other nanomaterials and represents a call to action for other research groups to monitor the toxicity of a nanomaterial concurrently with its development.
