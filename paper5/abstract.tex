\justifying
Research effort in membrane technology is divided into two fields depending on the material used: i) ceramic and ii) polymeric membranes. Motivated by the aim to harmonize the two research fields and to create solutions for advanced wastewater treatment, we investigated the possibility to originate a new class of membranes uniting the advantages of polymer and ceramic membranes. In particular, a Fully Carbon Membrane (FCM), constituted by hierarchical elemental carbon structures from the mechanical support to the selective layer, was manufactured. The FCM were characterized by a combination of surface science tools, capillary flow porometry studies, and thermogravimetric analyses, which reveal the ability of the FCM to resist to harsh cleaning by hypochlorite solutions and annealing cycles (typical of ceramic membranes). The FCM permeability and rejection performance were evaluated in a cross-flow setup and confirm the operation of the membrane in the nanofiltration regime (typical of polymeric membranes). In summary, the aspiration here is to initiate a new line of research in hybrid elemental carbon membrane materials that will support technologies for sustainable water resource use through wastewater reclamation.
