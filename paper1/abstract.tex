\justifying
Here we report the synthesis of graphene oxide nanoscrolls (GONS) with tunable dimensions via a high frequency ultrasound solution processing technique. GONS can be visualized as a graphene oxide (GO) sheet rolled into a spiral-wound structure and represent an alternative to traditional carbon nano-morphologies. The scrolling process is initiated by the ultrasound treatment which provides the scrolling activation energy for the formation of GONS. The GO and GONS dimensions are observed to be a function of ultrasound frequency, power density, and irradiation time. Ultrasonication increases GO and GONS C-C bonding likely due to in situ thermal reduction at the cavitating bubble-water interface. The GO area and GONS length are governed by two mechanisms; rapid oxygen defect site cleavage and slow cavitation mediated scission. Structural characterization indicates that GONS with tube and cone geometries can be formed with both narrow and wide dimensions in an industrial-scale time window. This work paves the way for GONS implementation for a variety of applications such as adsorptive and capacitive processes.