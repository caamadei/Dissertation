\justifying
Although graphene oxide (GO) has been widely used in a variety of research fields, the potential for GO implementation remains controversial. Researchers commonly define GO as a 2D carbon nanomaterial with oxygen functionalities, but this definition is too loose and leads the community to compare results among significantly different nanomaterials. In order to overcome this challenge, here we suggest high-throughput post-processing GO characterization techniques to rapidly and thoroughly define GO chemo-morphological properties. Then, based on characterization analysis and a clustering algorithm, we classified GO into six categories. The classification method was validated with GO samples obtained from different producers. The commercial samples were individually implemented to fabricate various macroscopic devices (\textit{e.g.}, membranes) and we observed that GO classified in the same category offered similar macroscopic performance. In contrast, samples from different categories resulted in a noticeable variation in macroscopic results, corroborating the importance of using standardized materials. The presented characterization and classification method will assist the research community by enabling a fair comparison between studies. Moreover, it will assist GO producers to target customers in a more-effective manner by distributing GO with optimal properties for a specific application, supporting the leap of GO from lab to industry.