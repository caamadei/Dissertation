The water crisis is one of the greatest challenges of our time.  Water is essential to agricultural production, food security, ecosystems; its scarcity jeopardizes the development of nations and basic human existence. The current water crisis is exacerbated by anthropogenic phenomena, such as climate change and increased water demand, that also undermine access to clean water.
Groundwater plays a pivotal role in the water supply. Almost half of the world’s population relies on this resource to satisfy basic needs. However, groundwater is extracted at faster rates than it is recharged by the natural cycle. New technological advances are needed to supply water in a more sustainable manner. One of these technologies is wastewater reclamation, in which wastewater is treated to a level that can be reused, thus avoiding the withdrawal of new freshwater from the natural cycles.
Innovation in wastewater reclamation relies on the use of new materials with more effective filtration capabilities. In particular, nanomaterials that exhibit special physical and chemical properties make them interesting products and a valid alternative to the currently used polymeric membranes. In recent years, researchers have focused on the use of graphene oxide (GO) in water treatment after molecular dynamic simulations and filtration experiments demonstrated promising features including permeability and the ability to sieve out ions and molecules. These remarkable results sparked a new line of research that has opened up new avenues of inquiry and scientific questions.
This dissertation attempts to address these scientific questions by elucidating the properties of GO and its performance as a molecular sieve. Three main results are contained in this work.
i.	Standardization of GO offering high-throughput characterization techniques, which facilitates discussions within the research community and comparisons of results.
ii.	Confirmation, using novel Lattice-Boltzmann simulations, of the friction-like water transport inside GO nanochannels. The simulations, supported by experimental permeation results, also identify the main GO properties affecting the water transport.
iii.	Evidence of the fine-tuning of GO properties, by creating GO nanoscrolls and fully carbon membranes. The latter material also shows the potential of GO to be used in nanofiltration applications and to be resistant to harsh chemical exposure.
This work ends with a concluding discussion aimed at establishing a new line of research based on the monitoring and control of possible GO toxicity based on its interactions with the human body.
