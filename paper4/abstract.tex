Active research in nanotechnology contemplates
the use of nanomaterials for environmental engineering
applications. However, a primary challenge is understanding
the effects of nanomaterial properties on industrial device
performance and translating unique nanoscale properties to the macroscale. One emerging example consists of graphene oxide (GO) membranes for separation processes. Thus, here we investigate how individual GO properties can impact GO
membrane characteristics and water permeability. GO chemistry and morphology were controlled with easy-to-implement photoreduction and sonication techniques and were quantitatively correlated, offering a valuable tool for accelerating characterization. Chemical GO modification allows for fine control of GO oxidation state, allowing control of GO architectural laminate (GOAL) spacing and permeability. Water permeability was measured for eight GOALs characterized by different GOAL chemistry and morphology and indicates that GOAL nanochannel height dictates water transport. The experimental outputs were corroborated with mesoscale water transport simulations of relatively large domains (thousands of square nanometers) and indicate a no-slip Darcy-like behavior inside the GOAL nanochannels. The experimental and simulation evidence presented in this study helps create a clearer picture of water transport in GOAL and can be used to rationally design more effective and efficient GO membranes.